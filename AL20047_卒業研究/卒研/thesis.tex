\documentclass[12pt,a4j]{jreport}
\setcounter{secnumdepth}{5}
\usepackage[dvipdfmx]{graphicx}
\usepackage{amsmath,amssymb}
\usepackage{comment}
\usepackage{graphicx}
\usepackage{here}
\usepackage{bm}
\usepackage{url}

\renewcommand{\baselinestretch}{1.5}

\renewcommand{\bibname}{参考文献}





\begin{document}


%%%%%%%%%%%%%%%%%%%%%%%%%%%%%%%%%%%%%
% 表紙
%%%%%%%%%%%%%%%%%%%%%%%%%%%%%%%%%%%%%
\begin{titlepage}

\begin{center}

    \vspace*{2cm}
    \Large 2023 年度 芝浦工業大学 工学部 情報工学科\\

    \vspace*{1.0cm}
    \Huge 卒 \qquad 業 \qquad 論 \qquad 文\\
    \vspace*{2.5cm}

    %TODO 編集 : 題目
    \Large StandarMLの開発環境向上のための関数の引数記述ミスの補完
    
    \vspace{4cm}
    \begin{tabular}{ll}
        %TODO 編集 : 題目
        \vspace*{2mm}
        学籍番号 & \qquad $\mathbf{AL20047}$ \\
        \vspace*{2mm}
        氏\phantom{  }名 & \qquad 庄司 \quad 歩夢   \\
        \vspace*{2mm}
        指導教員           & \qquad 篠埜 \quad 功
    \end{tabular}
\end{center}
\end{titlepage}




{\makeatletter
\let\ps@jpl@in\ps@empty
\makeatother
\pagestyle{empty}
\tableofcontents
\clearpage}

\setcounter{page}{1} 
\pagestyle{plain}

%%%%%%%%%%%%%%%%%%%%%%%%%%%%%%%%%%%%%%%%%%%%%%%%%%%%%%%%%%%%%
% 序論 
%%%%%%%%%%%%%%%%%%%%%%%%%%%%%%%%%%%%%%%%%%%%%%%%%%%%%%%%%%%%%
\chapter{はじめに}

\section{背景}
コード補完機能を利用することで,プログラム開発を効率的に行うことができる.コード補完とは,統合開発環境(IDE)などのコードエディタから利用可能な支援機能で,入力途中のコードに対してその続きを入力する候補を自動表示する機能である\cite{chubati}.この機能を利用することによって開発者は必なコードをすべて直接記述する必要がなくなるため,開発の効率を向上させることができる.

しかし,すべてのプログラミング言語に対してこのような機能が十分に整っていないのが現状である.実際に,関数型言語のStandard ML\cite{StandardML}はパターンマッチングによって複雑な条件分岐の記述を簡潔に書くことができたり,コンパイル時に型が検査され安全性が高いなどの特徴がありながらも,補完機能が充実していない.

また,三浦\cite{eff}はコード補完の有意性について調査し,補完機能を利用することで括弧の入力数が有意に下がり,補完機能の有用性を示した.よって本研究ではStandard MLに対して,関数の引数の括弧の記述忘れの補完について統合開発環境VScode\cite{vscode}で実装していく.



%%%%%%%%%%%%%%%%%

%[第一章コメント]

%課題が抽象的すぎるので課題をもっと具体的に<-論文を入れながらなので時間がかかるので実装と並行しつつ調べながらやる
%->現在書いてある課題は第二段落目の「補完機能が充実していない.」のみ
%-> x, y, z => (x, y, z)のように具体例を書き込むのも可


%%%%%%%%%%%%%%%%%


\section{本論文の構成}
本論文の構成は次の通りである.まず2章において関連研究について述べる.3章において補完機能の実装概要についてのべる.4章においてに字句解析手法ついて述べる.5章において構文解析手法について述べる.6章において補完候補の取得について述べる.7章において実装方法について述べる.8章において考察を述べる.9章においてまとめと今後の課題を述べる.

%%%%%%%%%%%%%%%%%%%%%%%%%%%%%%%%%%%%%%%%%%%%%%%%%%%%%%%%%%%%%
%%%%%%%%%%%%%%%%%%%%%%%%%%%%%%%%%%%%%%%%%%%%%%%%%%%%%%%%%%%%%
\chapter{関連研究}

\section{グループ1}
初学者が環境構築にかかる手間をブラウザのみで利用できるWeb IDEを利用することによって軽減し,プログラミング学習を進めるうえで,タイピングスキルによって影響が出ないようにするための手法として自動補完の実装が行われた\cite{eff}.三浦は,効率的なプログラミングのために一般的に導入されている自動補完機能に着目し,初学者がどのように利用するかを調査した.その結果,自動補完を活用する学習者とそうでない学者に二分された.利用頻度の高い補完候補には,for文やif文,画面描画関連の関数呼び出し,およびvoidであった.for文やif文のスニペットには、括弧やカーリーブラケットを含んでいたため,補完機能を利用した編集行為の前後の文と,補完機能を利用しなかった編集行為の前ぼの文と比較すると,括弧やカーリーブラケットの入力数が有意に低かった.


\section{グループ2}
先行研究\cite{sato}では,Standard MLのサブセット言語においてerrorトークンを挿入することで構文誤りが存在しても補完が行われる実装が行われた.実際に統合開発環境であるVScodeでStandard MLのサブセットに対して関数宣言による括弧の記述忘れと引数の記述忘れに対して変数名の補完が行われた.
また,福本ら\cite{fjei}は,errorトークンの数と補完精度についての議論が行われた.


%%%%%%%%%%%%%%%%%%%%%%%%%%%%%%%%%%%%%%%%%%%%%%%%%%%%%%%%%%%%%
%%%%%%%%%%%%%%%%%%%%%%%%%%%%%%%%%%%%%%%%%%%%%%%%%%%%%%%%%%%%%
\chapter{補完機能の実装概要}





%%%%%%%%%%%%%%%%%%%%%%%%%%%%%%%%%%%%%%%%%%%%%%%%%%%%%%%%%%%%%
%%%%%%%%%%%%%%%%%%%%%%%%%%%%%%%%%%%%%%%%%%%%%%%%%%%%%%%%%%%%%
\chapter{字句解析手法}
字句解析器を生成するプログラムであるFlex\cite{flex}というツールを用いて字句解析を行う.





%%%%%%%%%%%%%%%%%%%%%%%%%%%%%%%%%%%%%%%%%%%%%%%%%%%%%%%%%%%%%
%%%%%%%%%%%%%%%%%%%%%%%%%%%%%%%%%%%%%%%%%%%%%%%%%%%%%%%%%%%%%
\chapter{構文解析手法}
構文解析器を生成するパーサージェネレーターの1つであるBison\cite{bison}を用いて実装を行う.





%%%%%%%%%%%%%%%%%%%%%%%%%%%%%%%%%%%%%%%%%%%%%%%%%%%%%%%%%%%%%
%%%%%%%%%%%%%%%%%%%%%%%%%%%%%%%%%%%%%%%%%%%%%%%%%%%%%%%%%%%%%
\chapter{補完候補の取得}
Language Serverで補完候補の計算を行う\cite{lsp}.



\chapter{実装}




\chapter{考察}




\chapter{まとめと今後の課題}










\chapter*{謝辞}
\addcontentsline{toc}{chapter}{謝辞}


\newpage
\bibliographystyle{junsrt}
\addcontentsline{toc}{chapter}{参考文献}
\bibliography{ref.bib}


\end{document}
